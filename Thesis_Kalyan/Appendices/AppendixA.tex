
\chapter{Appendix}
\label{AppendixA}
\lhead{Appendix A. \emph{Appendix}}


\section{SMD Suite}

Sinha-Malo-Deb~(SMD) suite~\cite{sinha2014test} is a scalable set of test problems which cover a wide range of difficulties associated with bilevel optimization, including non-linearity, multi-modality and conflict in upper and lower level objectives. The suite contains twelve problems of which SMD1-SMD8 are unconstrained, whereas the rest are constrained problems. At each level, the problem contains three components, and the overall objective is calculated as the summation of these three components. The generic form of the objective functions is as shown in Equation~\ref{basic_1_ch2_app}.

\begin{equation}
\begin{array}{lr}
 \underset{\mathbf{x}_u}{\text{Minimize}} \hspace{2mm} F_u(\mathbf{x}_u, \mathbf{ x}_l)=$$F_1(\mathbf{x}_{u1})+F_2(\mathbf{x}_{l1}) +F_3(\mathbf{x}_{u2}, \mathbf{x}_{l2})$$\\
\hspace{9mm}\underset{\mathbf{x}_l}{\text{Minimize}}\hspace{2mm} f_l(\mathbf{x}_u, \mathbf{ x}_l)=$$f_1( \mathbf{x}_{u1}, \mathbf{x}_{u2})+f_2(\mathbf{x}_{l1}) +f_3(\mathbf{x}_{u2}, \mathbf{x}_{l2})$$,\\
\hspace{9mm} \text {where}\hspace{2mm}\mathbf{x}_u=( \mathbf{x}_{u1}, \mathbf{x}_{u2}), \hspace{2mm} \text{and} \hspace{2mm} \mathbf{x}_l=( \mathbf{x}_{l1}, \mathbf{x}_{l2})\\
\end{array}
\label{basic_1_ch2_app}
\end{equation}

Here $F_1$, $F_2$, $F_3$ are the subcomponents of the upper level objective function. Similarly, $f_1$, $f_2$, $f_3$ are treated as subcomponents of the lower level task. The dimension of the variables are set to $|\mathbf{x}_{u1}|=p$, $| \mathbf{x}_{u2}|=r$, $|\mathbf{x}_{l1}|=q$, $| \mathbf{x}_{l2}|=r$. For SMD6, $|\mathbf{x}_{l1}|=q+s$. The value p=1, q=2, r=1 is set for all problems except SMD6 where p=1, q=0, r=1, s=2. These settings correspond to all problems having five variables, two at upper and three at lower level for the studies presented.  

\subsection{SMD1}

The problem is described in the Equation~\ref{smd_1}. For this problem, the upper and lower levels are \emph{cooperative}. The lower level contains convex optimization task, whereas the upper level task is convex with respect to upper level variables and optimum lower level variables. 

\begin{equation}
\begin{array}{lr}
\text{}\hspace{2mm} F_1=$$\sum_{	i=1}^{p} x_{u1i}^{2} $$\\
\text{}\hspace{2mm} F_2=$$\sum_{	i=1}^{q} x_{l1i}^{2} $$\\
\text{}\hspace{2mm} F_3=$$\sum_{	i=1}^{r} x_{u2i}^{2} $$+$$\sum_{	i=1}^{r} (x_{u2i}-tan x_{l2i})^{2} $$\\

\text{}\hspace{2mm} f_1=$$\sum_{	i=1}^{p} x_{u1i}^{2} $$\\
\text{}\hspace{2mm} f_2=$$\sum_{	i=1}^{q} x_{l1i}^{2} $$\\
\text{}\hspace{2mm} f_3=$$\sum_{	i=1}^{r} (x_{u2i}-tan x_{l2i})^{2} $$\\

\end{array}
\label{smd_1}
\end{equation}

The ranges of the variables are as follows:
\begin{equation}
\begin{array}{lr}


\hspace{2mm}    $$x_{u1i} \in  [-5,10],          \forall i \in {1,2\ldots,p}$$\\
\hspace{2mm}	$$x_{u2i} \in  [-5,10],          \forall i \in {1,2\ldots,r}$$\\
\hspace{2mm}		$$x_{l1i} \in  [-5,10],          \forall i \in {1,2\ldots,q}$$\\
\hspace{2mm}	$$x_{l2i} \in  (\frac{-\pi}{2},\frac{\pi}{2}),          \forall i \in {1,2\ldots,r}$$
\end{array}
\label{smd_eqn1}
\end{equation}

The upper and lower level functions are 0 at the optimum. The optimum values of the upper level variables $\mathbf{x}_u=0$ and lower level variables $\mathbf{x}_l$ are given by the relationship: 

\begin{equation}
\begin{array}{lr}
\text{}\hspace{2mm} x_{l1i}=0,  \forall i \in {1,2\ldots,q} \\
\text{}\hspace{2mm} x_{l2i}=\tan^{-1} x_{u2i},  \forall i \in {1,2\ldots,r} \\
\end{array}
\label{basic_1}
\end{equation}

\subsection{SMD2}

This problem is similar to SMD1 with in terms of being convex at both levels. However the upper and lower level problems are conflicting instead of cooperative, which implies that an inaccurate lower level optima may lead to upper level value better than the true optimum. The problem is given in Equation~\ref{smd_2}.

\begin{equation}
\begin{array}{lr}
\text{}\hspace{2mm} F_1=$$\sum_{	i=1}^{p} x_{u1i}^{2} $$\\
\text{}\hspace{2mm} F_2=-$$\sum_{	i=1}^{q} x_{l1i}^{2} $$\\
\text{}\hspace{2mm} F_3=$$\sum_{	i=1}^{r} x_{u2i}^{2} $$-$$\sum_{	i=1}^{r} (x_{u2i}-log x_{l2i})^{2} $$\\

\text{}\hspace{2mm} f_1=$$\sum_{	i=1}^{p} x_{u1i}^{2} $$\\
\text{}\hspace{2mm} f_2=$$\sum_{	i=1}^{q} x_{l1i}^{2} $$\\
\text{}\hspace{2mm} f_3=$$\sum_{	i=1}^{r} (x_{u2i}-log x_{l2i})^{2} $$\\

\end{array}
\label{smd_2}
\end{equation}


The ranges of the variables are as follows:
\begin{equation}
\begin{array}{lr}

$$x_{u1i} \in  [-5,10],          \forall i \in {1,2\ldots,p}$$\\
$$x_{u2i} \in  [-5,1],          \forall i \in {1,2\ldots,r}$$\\
$$x_{l1i} \in  [-5,10],          \forall i \in {1,2\ldots,q}$$\\
$$x_{l2i} \in  (0,e],          \forall i \in {1,2\ldots,r}$$
\end{array}
\label{smd_equ2}
\end{equation}


The upper and lower level functions are 0 at the optimum. The optimum values of the upper level variables $\mathbf{x}_u=0$ and lower level variables $\mathbf{x}_l$ are given by the relationship: 

\begin{equation}
\begin{array}{lr}
\text{}\hspace{2mm} x_{l1i}=0,  \forall i \in {1,2\ldots,q} \\
\text{}\hspace{2mm} x_{l2i}=\log^{-1} x_{u2i},  \forall i \in {1,2\ldots,r} \\
\end{array}
\label{basic_2}
\end{equation}

\subsection{SMD3}

The two levels are cooperative for this problem. The difficulty is introduced in terms of multi-modality at the lower level function. The upper level is convex in terms of upper and optimum lower level variables. The constituent of the functions are as shown in Equation~\ref{smd_3}:


\begin{equation}
\begin{array}{lr}
\text{}\hspace{2mm} F_1=$$\sum_{	i=1}^{p} x_{u1i}^{2} $$\\
\text{}\hspace{2mm} F_2=$$\sum_{	i=1}^{q} x_{l1i}^{2} $$\\
\text{}\hspace{2mm} F_3=$$\sum_{	i=1}^{r} x_{u2i}^{2} $$+$$\sum_{	i=1}^{r} (x_{u2i}-tan x_{l2i})^{2} $$\\
\text{}\hspace{2mm} f_1=$$\sum_{	i=1}^{p} x_{u1i}^{2} $$\\
\text{}\hspace{2mm} f_2=q+$$\sum_{	i=1}^{q} (x_{l1i}^{2}-cos2 \pi x_{l1i} )$$\\
\text{}\hspace{2mm} f_3=$$\sum_{	i=1}^{r} (x_{u2i}-tan x_{l2i})^{2} $$\\

\end{array}
\label{smd_3}
\end{equation}



The ranges of the variables are as follows:
\begin{equation}
\begin{array}{lr}
$$x_{u1i} \in  [-5,10],          \forall i \in {1,2\ldots,p}$$\\
$$x_{u2i} \in  [-5,10],          \forall i \in {1,2\ldots,r}$$\\
$$x_{l1i} \in  [-5,10],          \forall i \in {1,2\ldots,q}$$\\
$$x_{l2i} \in  (\frac{-\pi}{2},\frac{\pi}{2}),          \forall i \in {1,2\ldots,r}$$
\end{array}
\label{smd_eqn3}
\end{equation}

The upper and lower level functions are 0 at the optimum. The optimum values of the upper level variables $\mathbf{x}_u=0$ and lower level variables $\mathbf{x}_l$ are given by the relationship: 

\begin{equation}
\begin{array}{lr}
\text{}\hspace{2mm} x_{l1i}=0,  \forall i \in {1,2\ldots,q} \\
\text{}\hspace{2mm} x_{l2i}=\tan^{-1} x_{u2i},  \forall i \in {1,2\ldots,r} \\
\end{array}
\label{basic_3}
\end{equation}

\subsection{SMD4}

The problem characteristics are somewhat similar to SMD3, except different multimodal functions are used at the lower level. The problem formulation is given in Equation~\ref{smd_4}.


\begin{equation}
\begin{array}{lr}
\text{}\hspace{2mm} F_1=$$\sum_{	i=1}^{p} x_{u1i}^{2} $$\\
\text{}\hspace{2mm} F_2=-$$\sum_{	i=1}^{q} x_{l1i}^{2} $$\\
\text{}\hspace{2mm} F_3=$$\sum_{	i=1}^{r} x_{u2i}^{2} $$-$$\sum_{	i=1}^{r} (|x_{u2i}|-log (1+x_{l2i}))^{2} $$\\
\text{}\hspace{2mm} f_1=$$\sum_{	i=1}^{p} x_{u1i}^{2} $$\\
\text{}\hspace{2mm} f_2=q+$$\sum_{	i=1}^{q} (x_{l1i}^{2}-cos2 \pi x_{l1i}) $$\\
\text{}\hspace{2mm} f_3=$$\sum_{	i=1}^{r} (|x_{u2i}|-log (1+x_{l2i}^{2}))^{2} $$\\

\end{array}
\label{smd_4}
\end{equation}



The ranges of the variables are as follows:
\begin{equation}
\begin{array}{lr}
$$x_{u1i} \in  [-5,10],          \forall i \in {1,2\ldots,p}$$\\
$$x_{u2i} \in  [-1,1],          \forall i \in {1,2\ldots,r}$$\\
$$x_{l1i} \in  [-5,10],          \forall i \in {1,2\ldots,q}$$\\
$$x_{l2i} \in  [0,e],          \forall i \in {1,2\ldots,r}$$
\end{array}
\label{smd_4_sub}
\end{equation}


The upper and lower level functions are 0 at the optimum. The optimum values of the upper level variables $\mathbf{x}_u=0$ and lower level variables $\mathbf{x}_l$ are given by the relationship: 

\begin{equation}
\begin{array}{lr}
\text{}\hspace{2mm} x_{l1i}=0,  \forall i \in {1,2\ldots,q} \\
\text{}\hspace{2mm} x_{l2i}=\log^{-1} |x_{u2i}| -1,  \forall i \in {1,2\ldots,r} \\
\end{array}
\label{basic_4}
\end{equation}


\subsection{SMD5}

This test problem is also has conflicting upper and lower levels. The global optimum at the lower level lies in a long, narrow, flat and parabolic valley as it has the Rosenbrock's (banana) function. The convexity of the upper level task depends on the upper level variables and optimum lower level variables. The constituents of the problem statement are given in Equation~\ref{smd_5}.
\begin{equation}
\begin{array}{lr}
\text{}\hspace{2mm} F_1=$$\sum_{	i=1}^{p} x_{u1i}^{2} $$\\
\text{}\hspace{2mm} F_2=-$$\sum_{	i=1}^{q-1} ((x_{l1i+1}-(x_{l1i})^{2}) +(x_{l1i}-1)^{2})$$\\
\text{}\hspace{2mm} F_3=$$\sum_{	i=1}^{r} x_{u2i}^{2} $$-$$\sum_{	i=1}^{r} (|x_{u2i}|-x_{l2i})^{2} $$\\

\text{}\hspace{2mm} f_1=$$\sum_{	i=1}^{p} x_{u1i}^{2} $$\\
\text{}\hspace{2mm} f_2=$$\sum_{	i=1}^{q-1} ((x_{l1i+1}-(x_{l1i})^{2}) +(x_{l1i}-1)^{2})$$\\
\text{}\hspace{2mm} f_3=$$\sum_{	i=1}^{r} (|x_{u2i}|-(x_{l2i})^{2})^{2} $$\\

\end{array}
\label{smd_5}
\end{equation}

The ranges of the variables are as follows:
\begin{equation}
\begin{array}{lr}
$$x_{u1i} \in  [-5,10],          \forall i \in {1,2\ldots,p}$$\\
$$x_{u2i} \in  [-5,10],          \forall i \in {1,2\ldots,r}$$\\
$$x_{l1i} \in  [-5,10],          \forall i \in {1,2\ldots,q}$$\\
$$x_{l2i} \in  [-5,10],          \forall i \in {1,2\ldots,r}$$
\end{array}
\label{smd_5_sub}
\end{equation}

The upper and lower level functions are 0 at the optimum. The optimum values of the upper level variables $\mathbf{x}_u=0$ and lower level variables $\mathbf{x}_l$ are given by the relationship: 

\begin{equation}
\begin{array}{lr}
\text{}\hspace{2mm} x_{l1i}=1,  \forall i \in {1,2\ldots,q} \\
\text{}\hspace{2mm} x_{l2i}={\sqrt(|x_{u2i}|)} ,  \forall i \in {1,2\ldots,r} \\
\end{array}
\label{basic_5}
\end{equation}


\subsection{SMD6}

The two levels are conflicting for this problem. For any given upper level vector the problem contains many globally optimum solutions at the lower level, making it more difficult to identify the ones that will result in global upper level optimum. The constituent functions are given in Equation~\ref{smd_6}.


\begin{equation}
\begin{array}{lr}
\text{}\hspace{2mm} F_1=$$\sum_{	i=1}^{p} x_{u1i}^{2} $$\\
\text{}\hspace{2mm} F_2=-$$\sum_{	i=1}^{q} x_{l1i}^{2} $$+$$\sum_{	i=q+1}^{q+s} x_{l1i}^{2} $$\\
\text{}\hspace{2mm} F_3=$$\sum_{	i=1}^{r} x_{u2i}^{2} $$-$$\sum_{	i=1}^{r} (x_{u2i}- x_{l2i})^{2} $$\\

\text{}\hspace{2mm} f_1=$$\sum_{	i=1}^{p} x_{u1i}^{2} $$\\
\text{}\hspace{2mm} f_2=$$\sum_{	i=1}^{q} x_{l1i}^{2} $$+$$\sum_{	i=q+1,i=i+2}^{q+s-1} (x_{l1i+1}-x_{l1i})^{2} $$\\
\text{}\hspace{2mm} f_3=$$\sum_{	i=1}^{r} (x_{u2i}- x_{l2i})^{2} $$\\

\end{array}
\label{smd_6}
\end{equation}


The ranges of the variables are as follows:
\begin{equation}
\begin{array}{lr}
$$x_{u1i} \in  [-5,10],          \forall i \in {1,2\ldots,p}$$\\
$$x_{u2i} \in  [-5,10],          \forall i \in {1,2\ldots,r}$$\\
$$x_{l1i} \in  [-5,10],          \forall i \in {1,2\ldots,q}$$\\
$$x_{l2i} \in  [-5,10],          \forall i \in {1,2\ldots,r}$$
\end{array}
\label{smd_6_sub}
\end{equation}


The upper and lower level functions are 0 at the optimum. The optimum values of the upper level variables $\mathbf{x}_u=0$ and lower level variables $\mathbf{x}_l$ are given by the relationship: 

\begin{equation}
\begin{array}{lr}
\text{}\hspace{2mm} x_{l1i}=0,  \forall i \in {1,2\ldots,q} \\
\text{}\hspace{2mm} x_{l2i}=x_{u2i} ,  \forall i \in {1,2\ldots,r} \\
\end{array}
\label{basic_6}
\end{equation}


\subsection{SMD7}

The complexity of upper level problem is higher for this problem compared to the previous ones. The constituten functions are given in Equation~\ref{smd_7}.

\begin{equation}
\begin{array}{lr}
\text{}\hspace{2mm} F_1=1+\frac{1}{400}$$\sum_{i=1}^{p} x_{u1i}^{2} $$-$$\prod_{i=1}^{p} cos \frac{x_{u1i}}{\sqrt{i}} $$\\
\text{}\hspace{2mm} F_2=-$$\sum_{	i=1}^{q} x_{l1i}^{2} $$\\
\text{}\hspace{2mm} F_3=$$\sum_{	i=1}^{r} x_{u2i}^{2} $$-$$\sum_{	i=1}^{r} (x_{u2i}-log x_{l2i})^{2} $$\\


\text{}\hspace{2mm} f_1=$$\sum_{	i=1}^{p} x_{u1i}^{3} $$\\
\text{}\hspace{2mm} f_2=$$\sum_{	i=1}^{q} x_{l1i}^{2} $$\\
\text{}\hspace{2mm} f_3=$$\sum_{	i=1}^{r} (x_{u2i}-log x_{l2i})^{2} $$\\

\end{array}
\label{smd_7}
\end{equation}


The ranges of the variables are as follows:
\begin{equation}
\begin{array}{lr}
$$x_{u1i} \in  [-5,10],          \forall i \in {1,2\ldots,p}$$\\
$$x_{u2i} \in  [-5,1],          \forall i \in {1,2\ldots,r}$$\\
$$x_{l1i} \in  [-5,10],          \forall i \in {1,2\ldots,q}$$\\
$$x_{l2i} \in  (0,e],          \forall i \in {1,2\ldots,r}$$
\end{array}
\label{smd_7_sub}
\end{equation}

The upper and lower level functions are 0 at the optimum. The optimum values of the upper level variables $\mathbf{x}_u=0$ and lower level variables $\mathbf{x}_l$ are given by the relationship: 

\begin{equation}
\begin{array}{lr}
\text{}\hspace{2mm} x_{l1i}=0,  \forall i \in {1,2\ldots,q} \\
\text{}\hspace{2mm} x_{l2i}=\log^{-1}x_{u2i} ,  \forall i \in {1,2\ldots,r} \\
\end{array}
\label{basic_7}
\end{equation}



\subsection{SMD8}

This test problem contains multi-modality~(Ackley's function) at the upper level and difficulty in convergence~(Rosenbrock's function) at lower level at the same time. The two levels are conflicting. The constituents of the problem are shown in Equation~\ref{smd_8}.

\begin{equation}
\begin{array}{lr}
$$\text{}\hspace{2mm} F_1=20+e-20 exp(-0.2 \sqrt{\frac{1}{p} \sum_{	i=1}^{p} (x_{u1i})^{2}}) -exp (\frac{1}{p}\sum_{	i=1}^{p} cos2 \pi x_{u1i}  )$$ \\
\text{}\hspace{2mm} F_2=-$$\sum_{	i=1}^{q-1} ((x_{l1ui+1}-(x_{l1i})^{2}) +(x_{l1i}-1)^{2})$$\\
\text{}\hspace{2mm} F_3=$$\sum_{	i=1}^{r} (x_{u2i})^{2} $$-$$\sum_{	i=1}^{r} (x_{u2i}- (x_{l2i})^{3})^{2} $$\\
\text{}\hspace{2mm} f_1=$$\sum_{	i=1}^{p} |x_{u1i}| $$\\
\text{}\hspace{2mm} f_2=$$\sum_{	i=1}^{q-1} ((x_{l1i+1}-(x_{l1i})^{2}) +(x_{l1i}-1)^{2})$$\\
\text{}\hspace{2mm} f_3=$$\sum_{	i=1}^{r} (x_{u2i}-(x_{l2i})^{3})^{2} $$\\
\end{array}
\label{smd_8}
\end{equation}

The ranges of the variables are as follows:
\begin{equation}
\begin{array}{lr}
$$x_{u1i} \in  [-5,10],          \forall i \in {1,2\ldots,p}$$\\
$$x_{u2i} \in  [-5,10],          \forall i \in {1,2\ldots,r}$$\\
$$x_{l1i} \in  [-5,10],          \forall i \in {1,2\ldots,q}$$\\
$$x_{l2i} \in  [-5,10],          \forall i \in {1,2\ldots,r}$$
\end{array}
\label{smd_8_sub}
\end{equation}

The upper and lower level functions are 0 at the optimum. The optimum values of the upper level variables $\mathbf{x}_u=0$ and lower level variables $\mathbf{x}_l$ are given by the relationship: 

\begin{equation}
\begin{array}{lr}
\text{}\hspace{2mm} x_{l1i}=0,  \forall i \in {1,2\ldots,q} \\
\text{}\hspace{2mm} x_{l2i}=(x_{u2i})^\frac{1}{3},  \forall i \in {1,2\ldots,r} \\
\end{array}
\label{basic_8}
\end{equation}


\subsection{SMD9}

This test problem has constraints at both upper and lower levels, defined in a way such that it causes convergence difficulties at both level independently. The constraints divide the search space into the annular regions and cause convergence difficulties without altering the global optima. The two levels are also conflicting in nature. The constituents functions are given in Equation~\ref{smd_9}.

\begin{equation}
\begin{array}{lr}
\text{}\hspace{2mm} F_1=$$\sum_{	i=1}^{p} (x_{u1i})^{2} $$\\
\text{}\hspace{2mm} F_2=-$$\sum_{	i=1}^{q} (x_{l1i})^{2} $$\\
\text{}\hspace{2mm} F_3=$$\sum_{	i=1}^{r} (x_{u2i})^{2} $$-$$\sum_{	i=1}^{r} (x_{u2}^{i}-log (1+x_{l2}^{i}))^{2} $$\\
\text{}\hspace{2mm} f_1=$$\sum_{	i=1}^{p} (x_{u1i})^{2} $$\\
\text{}\hspace{2mm} f_2=$$\sum_{	i=1}^{q} (x_{l1i})^{2} $$\\
\text{}\hspace{2mm} f_3=$$\sum_{	i=1}^{r} (x_{u2i}-log (1+x_{l2i}))^{2} $$\\
\end{array}
\label{smd_9}
\end{equation}

The upper and lower level constraints are given as below:

\begin{equation}
\begin{array}{lr}
\text{}\hspace{2mm} {G_1~(UL):} {\frac{\sum_{i=1}^{p} (x_{u1i})^{2} +\sum_{	i=1}^{r} (x_{u2i})^{2} }{a}} - \floor*{{{\frac{\sum_{	i=1}^{p} (x_{u1i})^{2} +\sum_{	i=1}^{r} (x_{u2i})^{2} }{b}} + {\frac{0.5}{b}}}} \geq 0\\
\text{}\hspace{2mm} {g_1~(LL):} {\frac{\sum_{	i=1}^{q} (x_{l1i})^{2} +\sum_{	i=1}^{r} (x_{l2i})^{2} }{a}} - \floor*{{{\frac{\sum_{	i=1}^{q} (x_{l1i})^{2} +\sum_{	i=1}^{r} (x_{l2i})^{2} }{b}} + {\frac{0.5}{b}}}} \geq 0
\end{array}
\label{smd_9_con}
\end{equation}

The ranges of the variables are as follows:
\begin{equation}
\begin{array}{lr}
$$x_{u1i} \in  [-5,10],          \forall i \in {1,2\ldots,p}$$\\
$$x_{u2i} \in  [-5,1],          \forall i \in {1,2\ldots,r}$$\\
$$x_{l1i} \in  [-5,10],          \forall i \in {1,2\ldots,q}$$\\
$$x_{l2i} \in  (-1,-1+e],          \forall i \in {1,2\ldots,r}$$
\end{array}
\label{smd_9_boun}
\end{equation}
The upper and lower level functions assume a value of 0 at the optimum. The optimum values of the upperl level variables are $\mathbf{x}_u=0$. The relationship between upper and lower level optimal variables is as follows: 

\begin{equation}
\begin{array}{lr}
\text{}\hspace{2mm} x_{l1i}=0,  \forall i \in {1,2\ldots,q} \\
\text{}\hspace{2mm} x_{l2i}=\log^{ - 1} x_{u2i}-1,  \forall i \in {1,2\ldots,r} \\
\end{array}
\label{basic_9}
\end{equation}

\subsection{SMD10}

For this problem, as the number of variables at the upper and lower level vary, the number of constraints also vary. In this case, the upper level consraints are functions of the upper level variables and the lower level constraints are as functions of the lower level variables. The constituent functions are given in Equation~\ref{smd_10_boun}.

\begin{equation}
\begin{array}{lr}
\text{}\hspace{2mm} F_1=$$\sum_{	i=1}^{p} (x_{u1i}-2)^{2} $$\\
\text{}\hspace{2mm} F_2=$$\sum_{	i=1}^{q} x_{l1i}^{2} $$\\
\text{}\hspace{2mm} F_3=$$\sum_{	i=1}^{r} (x_{u2i}-2)^{2} $$-$$\sum_{	i=1}^{r} (x_{u2i}-tan (x_{l2i}))^{2} $$\\

\text{}\hspace{2mm} f_1=$$\sum_{	i=1}^{p} x_{u1i}^{2} $$\\
\text{}\hspace{2mm} f_2=$$\sum_{	i=1}^{q} (x_{l1i}-2)^{2} $$\\
\text{}\hspace{2mm} f_3=$$\sum_{	i=1}^{r} (x_{u2i}-tan (x_{l2i}))^{2} $$\\

\end{array}
\label{smd_10_boun}
\end{equation}


The upper and lower level constraints are given as below:

\begin{equation}
\begin{array}{lr}
\text{}\hspace{2mm} {G_i~(UL):} x_{u1i}+ (x_{u1i})^{3} - \sum_{i=1}^{p} (x_{u1i})^{3} -\sum_{	i=1}^{r} (x_{u2i})^{3}\geq 0,~\forall i \in {1,2\ldots,p} \\
\text{}\hspace{2mm} {G_{p+i}~(UL):} x_{u2i}+ (x_{u2i})^{3} - \sum_{i=1}^{r} (x_{u2i})^{3} -\sum_{	i=1}^{p} (x_{u1i})^{3}\geq 0,~\forall i \in {1,2\ldots,r} \\

\text{}\hspace{2mm} {g_i~(LL):} x_{l1i}+ (x_{l1i})^{3} - \sum_{i=1}^{q} (x_{l1i})^{3} \geq 0,~\forall i \in {1,2\ldots,q} \\
\end{array}
\label{smd_10_con}
\end{equation}

The ranges of the variables are as follows:
\begin{equation}
\begin{array}{lr}
$$x_{u1i} \in  [-5,10],          \forall i \in {1,2\ldots,p}$$\\
$$x_{u2i} \in  [-5,10],          \forall i \in {1,2\ldots,r}$$\\
$$x_{l1i} \in  [-5,10],          \forall i \in {1,2\ldots,q}$$\\
$$x_{l2i} \in  (\frac{-\pi}{2},\frac{\pi}{2}),          \forall i \in {1,2\ldots,r}$$
\end{array}
\label{smd_10}
\end{equation}



The upper and lower level functions are $F_u=4$ and $f_l=3$  at the optimum. The optimum values of the upper level variables $\mathbf{x}_u=\frac{1}{\sqrt{p+r-1}}$ and lower level variables $\mathbf{x}_l$ are given by the relationship: 

\begin{equation}
\begin{array}{lr}
\text{}\hspace{2mm} x_{l1i}=\frac{1}{\sqrt{q-1}},  \forall i \in {1,2\ldots,q} \\
\text{}\hspace{2mm} x_{l2i}=\tan^{-1} x_{u2i} ,  \forall i \in {1,2\ldots,r} \\
\end{array}
\label{basic_10}
\end{equation}



\subsection{SMD11}

The lower level has one constraint for this problem, while th enumber of upper level constraints depend on the number of variables. For a given upper level vector the lower level has multiple globally optimum solutions. Both the lower and upper level constraints are active at the true optimum. The constituent functions are given in Equation~\ref{smd_11}.

\begin{equation}
\begin{array}{lr}
\text{}\hspace{2mm} F_1=$$\sum_{	i=1}^{p} (x_{u1i})^{2} $$\\
\text{}\hspace{2mm} F_2=-$$\sum_{	i=1}^{q} x_{l1i}^{2} $$\\
\text{}\hspace{2mm} F_3=$$\sum_{	i=1}^{r} (x_{u2i})^{2} $$-$$\sum_{	i=1}^{r} (x_{u2i}-log (x_{l2i}))^{2} $$\\

\text{}\hspace{2mm} f_1=$$\sum_{	i=1}^{p} x_{u1i}^{2} $$\\
\text{}\hspace{2mm} f_2=$$\sum_{	i=1}^{q} (x_{l1i})^{2} $$\\
\text{}\hspace{2mm} f_3=$$\sum_{	i=1}^{r} (x_{u2i}-log (x_{l2i}))^{2} $$\\

\end{array}
\label{smd_11}
\end{equation}


The upper and lower level constraints are given as below:

\begin{equation}
\begin{array}{lr}
\text{}\hspace{2mm} {G_i~(UL):} x_{u2i} \geq \frac{1}{\sqrt{r}}+\log x_{l2i} ,~\forall i \in {1,2\ldots,r} \\


\text{}\hspace{2mm} {g_1~(LL):} \sum_{i=1}^{r} (x_{u2i}- \log x_{l2i})^2 \geq 1,~\forall i \in {1,2\ldots,r} \\
\end{array}
\label{smd_11_con}
\end{equation}



The ranges of the variables are as follows:
\begin{equation}
\begin{array}{lr}
$$x_{u1i} \in  [-5,10],          \forall i \in {1,2\ldots,p}$$\\
$$x_{u2i} \in  [-1,1],          \forall i \in {1,2\ldots,r}$$\\
$$x_{l1i} \in  [-5,10],          \forall i \in {1,2\ldots,q}$$\\
$$x_{l2i} \in  (\frac{1}{e},e],          \forall i \in {1,2\ldots,r}$$
\end{array}
\label{smd_11_boun}
\end{equation}

The upper and lower level functions are $F_u=-1$ and $f_l=1$  at the optimum. The optimum values of the upper level variables $\mathbf{x}_{u}=0$ and lower level variables $\mathbf{x}_l$ are given by the relationship: 

\begin{equation}
\begin{array}{lr}
\text{}\hspace{2mm} x_{l1i}=0,  \forall i \in {1,2\ldots,q} \\
\text{}\hspace{2mm} x_{l2i}=\log^{-1} \frac{-1}{\sqrt{r}} ,  \forall i \in {1,2\ldots,r} \\
\end{array}
\label{basic_11}
\end{equation}



\subsection{SMD12}

For this problem, the number of constraints at both the levels increase as the number of variables are increased. For a given upper level vector, the lower level contains multiple global optima. All lower level constraints are active at the true optimum. The constituent functions are shown in Equation~\ref{smd_12}.


\begin{equation}
\begin{array}{lr}
\text{}\hspace{2mm} F_1=$$\sum_{	i=1}^{p} (x_{u1i}-2)^{2} $$\\
\text{}\hspace{2mm} F_2=$$\sum_{	i=1}^{q} x_{l1i}^{2} $$\\
\text{}\hspace{2mm} F_3=$$\sum_{	i=1}^{r} (x_{u2i}-2)^{2} $$-$$\sum_{	i=1}^{r} (x_{u2i}-tan (x_{l2i}))^{2} $$\\

\text{}\hspace{2mm} f_1=$$\sum_{	i=1}^{p} x_{u1i}^{2} $$\\
\text{}\hspace{2mm} f_2=$$\sum_{	i=1}^{q} (x_{l1i}-2)^{2} $$\\
\text{}\hspace{2mm} f_3=$$\sum_{	i=1}^{r} (x_{u2i}-tan (x_{l2i}))^{2} $$\\
\end{array}
\label{smd_12}
\end{equation}

The upper and lower level constraints are given as below:

\begin{equation}
\begin{array}{lr}
\text{}\hspace{2mm} {G_i~(UL):} x_{u1i}+ (x_{u1i})^{3} - \sum_{i=1}^{p} (x_{u1i})^{3} -\sum_{	i=1}^{r} (x_{u2i})^{3}\geq 0,~\forall i \in {1,2\ldots,p} \\
\text{}\hspace{2mm} {G_{p+i}~(UL):} x_{u2i}+ (x_{u2i})^{3} - \sum_{i=1}^{r} (x_{u2i})^{3} -\sum_{	i=1}^{p} (x_{u1i})^{3}\geq 0,~\forall i \in {1,2\ldots,r} \\

\text{}\hspace{2mm} {G_{p+r+j}~(UL):} x_{u2j}- \tan x_{l2j} \geq 0,~\forall j \in {1,2\ldots,r} \\


\text{}\hspace{2mm} {g_i~(LL):} x_{l1i}+ (x_{l1i})^{3} - \sum_{i=1}^{q} (x_{l1i})^{3} \geq 0,~\forall i \in {1,2\ldots,q} \\
\text{}\hspace{2mm} {g_{q+r}~(LL):}  \sum_{i=1}^{r} (x_{u2i}-\tan x_{l2i})^{2} \geq 0,~\forall i \in {1,2\ldots,r} \\

\end{array}
\label{smd_12_con}
\end{equation}


The ranges of the variables are as follows:
\begin{equation}
\begin{array}{lr}
$$x_{u1i} \in  [-5,10],          \forall i \in {1,2\ldots,p}$$\\
$$x_{u2i} \in  [-14.10,14.10],          \forall i \in {1,2\ldots,r}$$\\
$$x_{l1i} \in  [-5,10],          \forall i \in {1,2\ldots,q}$$\\
$$x_{l2i} \in  (-1.5,1.5),          \forall i \in {1,2\ldots,r}$$
\end{array}
\label{smd_12_boun}
\end{equation}	


The upper and lower level functions are $F_u=3$ and $f_l=4$  at the optimum. The values of the variables at the optimum are $\mathbf{x}_{u1}=\frac{1}{\sqrt{p+r-1}}$, $\mathbf{x}_{u2}=\frac{1}{\sqrt{p+r-1}}$, $\mathbf{x}_{l1}=\frac{1}{\sqrt{q-1}}$, and $\mathbf{x}_{l2}=\tan^{-1}(\frac{1}{\sqrt{p+r-1}}-\frac{1}{\sqrt{r}})$



\section{BLTP Suite}

The second set of thirteen test problems considered here are collected from various references in the literature~\cite{Angelobilevel,bard1988convex,rajesh2003tabu,aiyoshi1984solution,bard1982explicit,candler1982linear,oduguwa2002bi}. The problems are well known in the bilevel optimization domain and have often been used to evaluate the efficacy of different bilevel optimization algorithms. Since a number of these problems have been solved with classical methods, they mostly involve linear and/or quadratic objective and constraint functions. For some of the problems~(BLTP1, BLTP3, BLTP5, BLTP12), the true optimum values are not known, and hence the best considered in the literature are listed. For ease of reference, the problems have been labelled here as BLTP1-BLTP13~(BLTP = Bilevel Test Problems).

%Angelo et.al~\cite{Angelobilevel} compared there algorithm with a bunch of bilevel optimization problems. This article refers some articles for bilevel problems among them~\cite{bard1988convex,rajesh2003tabu,aiyoshi1984solution,bard1982explicit,candler1982linear} etc are well known to bilevel community and analysed bilevel problems and algorithms elaborately and efficiently. We collect most of the test problems from there. Others are taken from~\cite{oduguwa2002bi}. Here, problem description is defined as $\mathbf{X}=(\mathbf{x}_u, \mathbf{x}_l)$. The dimension of the $\mathbf{x}_u$ and $\mathbf{x}_l$ are varies problem to problem. The source of the problems and short analysis are given below

\subsection{BLTP1}

This problem is taken from~\cite{aiyoshi1984solution}. The formulation is described in Equation~\ref{bltp_1}. 

%This is a linear bilevel problem with multiple constraints at the lower level and a single constraint at the upper level. The referenced article solve this problem by using penalty method. The constituent of the problem statement is shown in the figure~\ref{bltp_1}.


\begin{equation}
\begin{array}{lr}
\underset{\mathbf{x}_u} {\text{Minimize}}\hspace{2mm} F_u(\mathbf{x}_u, \mathbf{ x}_l)=2x_{u1}+2x_{u2}-3x_{l1}-3x_{l2}-60\\

\hspace{16mm}s.t\hspace{2mm} x_{u1}+x_{u2}+x_{l1}-2x_{l2}-40\leq 0\\

\hspace{9mm} \underset{\mathbf{x}_l} {\text{Minimize}}\hspace{2mm} f_l(\mathbf{x}_u, \mathbf{ x}_l)=(x_{l1}-x_{u1}+20)^2+(x_{l2}-x_{u2}+20)^2 \\

\hspace{25mm}s.t\hspace{2mm} 2x_{l1}-x_{u1}+10\leq 0\\

\hspace{25mm}  2x_{l2}-x_{u2}+10 \leq 0\\

\hspace{25mm} \mathbf{x}_u \in {[0,50]},\mathbf{x}_l \in [-10,20] \\
\end{array}
\label{bltp_1}
\end{equation}

The best known values $F_u^*$=0 and  $f_l^*$=200 are reported in~\cite{Angelobilevel}. The corresponding variable values are $(\mathbf{x}^*_u, \mathbf{x}^*_l)=(0,0,-10,-10).$



\subsection{BLTP2}

This problem is taken from~\cite{bard1988convex}. The formulation is described in Equation~\ref{bltp_2}. 

\begin{equation}
\begin{array}{lr}
\underset{\mathbf{x}_u}{\text{Minimize}}\hspace{2mm} F_u(\mathbf{x}_u, \mathbf{ x}_l)=(x_u-5)^2+(2x_l+1)^2\\

\hspace{9mm} \underset{\mathbf{x}_l} {\text{Minimize}}\hspace{2mm} f_l(\mathbf{x}_u, \mathbf{ x}_l)=(x_l-1)^2-1.5 x_u x_l \\
\hspace{25mm}s.t\hspace{2mm} 3x_u-x_l \geq 3\\
\hspace{25mm} -x_u+0.5x_l \geq -4\\
\hspace{25mm} -x_u-x_l \geq -7\\

\hspace{25mm} \mathbf{x}_u \in {[0,2]},\mathbf{x}_l \in [0,1] \\
%\hspace{25mm} \mathbf{x}_u, \mathbf{ x}_l \geq 0\\
\end{array}
\label{bltp_2}
\end{equation}

The optimum values $F_u^*$=17 and  $f_l^*$=1 are reported in ~\cite{Angelobilevel}. The corresponding optimum variables are $(\mathbf{x}^*_u, \mathbf{x}^*_l)=(1,0).$

\subsection{BLTP3}

This problem is taken from~\cite{oduguwa2002bi,Angelobilevel}. The formulation is described in Equation~\ref{bltp_3}. 

\begin{equation}
\begin{array}{lr}
\underset{\mathbf{x}_u}{\text{Minimize}}\hspace{2mm} F_u(\mathbf{x}_u, \mathbf{ x}_l)=(x_u-1)^2+(x_l-1)^2\\

\hspace{9mm} \underset{\mathbf{x}_l} {\text{Minimize}}\hspace{2mm} f_l(\mathbf{x}_u, \mathbf{ x}_l)=0.5x_l^2+500x_l-50 x_u x_l \\
\hspace{25mm} \mathbf{x}_u \in {[0,1]},\mathbf{x}_l \in [0,1] \\

%\hspace{25mm} \mathbf{x}_u, \mathbf{ x}_l \geq 0\\
\end{array}
\label{bltp_3}
\end{equation}

The best known values $F_u^*$=1 and  $f_l^*$=0 are reported in ~\cite{Angelobilevel}. The corresponding variables are $(\mathbf{x}^*_u, \mathbf{x}^*_l)=(1,0).$

\subsection{BLTP4}

This problem is taken from~\cite{oduguwa2002bi}. The formulation is described in Equation~\ref{bltp_4}. 

\begin{equation}
\begin{array}{lr}
\underset{\mathbf{x}_u}{\text{Maximize}}\hspace{2mm} F_u(\mathbf{x}_u, \mathbf{ x}_l)=100x_u+1000x_{l1}\\


\hspace{9mm} \underset{\mathbf{x}_l} {\text{Maximize}}\hspace{2mm} f_l(\mathbf{x}_u, \mathbf{ x}_l)=x_{l1}+x_{l2} \\

\hspace{25mm}s.t\hspace{2mm} x_u+x_{l1}-x_{l2}\leq 1\\
\hspace{25mm}  x_{l1}+x_{l2} \leq 1\\
\hspace{25mm} \mathbf{x}_u \in {[0,1]},\mathbf{x}_l \in {[0,1]} \\
\end{array}
\label{bltp_4}
\end{equation}

The optimum values $F_u^*$=1000 and  $f_l^*$=1 are reported in ~\cite{Angelobilevel}. The corresponding optimum variables are $(\mathbf{x}^*_u, \mathbf{x}^*_l)=(1,0,1).$


\subsection{BLTP5}

This problem is taken from~\cite{savard1994steepest}. The formulation is described in Equation~\ref{bltp_5}. 

\begin{equation}
\begin{array}{lr}
\underset{\mathbf{x}_u}{\text{Minimize}}\hspace{2mm} F_u(\mathbf{x}_u, \mathbf{ x}_l)=(x_u-1)^2+2x_{l1}^2-2x_u\\


\hspace{9mm} \underset{\mathbf{x}_l} {\text{Minimize}}\hspace{2mm} f_l(\mathbf{x}_u, \mathbf{ x}_l)=(2x_{l1}-4)^2+(2x_{l2}-1)^2+x_u x_{l1} \\
\hspace{25mm}s.t\hspace{2mm} 4x_u+5x_{l1}+4x_{l2}\leq 12\\
\hspace{25mm}  -4x_u-5x_{l1}+4x_{l2} \leq -4\\
\hspace{25mm}  4x_u-4x_{l1}+5x_{l2} \leq 4\\
\hspace{25mm}  -4x_u+4x_{l1}+5x_{l2} \leq 4\\
\hspace{25mm} \mathbf{x}_u \in {[0,4]},\mathbf{x}_l \in [0,2] \\
%\hspace{25mm} \mathbf{x}_u, \mathbf{x}_l \geq 0 \\
\end{array}
\label{bltp_5}
\end{equation}

The best known values $F_u^*$=-1.209876568 and  $f_l^*$=7.61728 are reported in ~\cite{Angelobilevel}. The corresponding optimum variables are $(\mathbf{x}^*_u, \mathbf{x}^*_l)=(1.88889,0.888889,0).$

\subsection{BLTP6}


This problem is taken from~\cite{oduguwa2002bi}. The formulation is described in Equation~\ref{bltp_6}. 

\begin{equation}
\begin{array}{lr}
\underset{\mathbf{x}_u}{\text{Minimize}}\hspace{2mm} F_u(\mathbf{x}_u, \mathbf{ x}_l)=x_u^2+(x_l-10)^2\\

\hspace{16mm}s.t\hspace{2mm} -x_u+x_l\leq 0\\

\hspace{9mm} \underset{\mathbf{x}_l} {\text{Minimize}}\hspace{2mm} f_l(\mathbf{x}_u, \mathbf{ x}_l)=(x_u+2x_l-30)^2\\

\hspace{25mm}s.t\hspace{2mm} x_u+x_l\leq 20\\

\hspace{25mm} \mathbf{x}_u \in {[0,15]},\mathbf{x}_l \in [0,20] \\
\end{array}
\label{bltp_6}
\end{equation}

The optimum values $F_u^*$=100 and  $f_l^*$=0 are reported in ~\cite{Angelobilevel}. The corresponding optimum variables are $(\mathbf{x}^*_u, \mathbf{x}^*_l)=(10,10).$

\subsection{BLTP7}

This problem is taken from~\cite{anandalingam1990solution}. The formulation is described in Equation~\ref{bltp_7}. 

\begin{equation}
\begin{array}{lr}
\underset{\mathbf{x}_u}{\text{Maximize}}\hspace{2mm} F_u(\mathbf{x}_u, \mathbf{ x}_l)=x_u+3x_l\\

\hspace{9mm} \underset{\mathbf{x}_l} {\text{Maximize}}\hspace{2mm} f_l(\mathbf{x}_u, \mathbf{ x}_l)=x_u-3x_l \\
\hspace{25mm}s.t\hspace{2mm} -x_u-2x_l\leq -10\\
\hspace{25mm} x_u-2x_l\leq 6\\
\hspace{25mm} 2x_u-x_l\leq 21\\
\hspace{25mm} x_u+2x_l\leq 38\\
\hspace{25mm} -x_u+2x_l\leq 18\\
\hspace{25mm} \mathbf{x}_u \in {[0,16]},\mathbf{x}_l \in [0,12] \\
%\hspace{25mm} \mathbf{x}_u, \mathbf{x}_l \geq 0\\
\end{array}
\label{bltp_7}
\end{equation}
The optimum values $F_u^*$=49 and $f_l^*$=-17 are reported in \cite{Angelobilevel}. The corresponding optimum variables are $(\mathbf{x}^*_u, \mathbf{x}^*_l)=(16,11).$

\subsection{BLTP8}

This problem is taken from~\cite{falk1995bilevel}. The formulation is described in Equation~\ref{bltp_8}. 

\begin{equation}
\begin{array}{lr}
\underset{\mathbf{x}_u}{\text{Minimize}}\hspace{2mm} F_u(\mathbf{x}_u, \mathbf{ x}_l)=x_{u1}^2-2x_{u1}+x_{u2}^2-2x_{u2}+x_{l1}^2+x_{l2}^2\\


\hspace{9mm} \underset{\mathbf{x}_l} {\text{Minimize}}\hspace{2mm} f_l(\mathbf{x}_u, \mathbf{ x}_l)=(x_{l1}-x_{u1})^2+(x_{l2}-x_{u2})^2 \\

\hspace{25mm} \mathbf{x}_u \in {[0,1.5]},\mathbf{x}_l \in [0.5,1.5] \\
%\hspace{25mm} \mathbf{x}_u \geq 0,\mathbf{x}_l \in [0.5,1.5] \\
\end{array}
\label{bltp_8}
\end{equation}

The optimum values $F_u^*$= -1 and  $f_l^*$=0 are reported in ~\cite{Angelobilevel}. The corresponding optimum variables are $(\mathbf{x}^*_u, \mathbf{x}^*_l)=(0.5,0.5,0.5,0.5).$

\subsection{BLTP9}

This problem is taken from~\cite{rajesh2003tabu}. The formulation is described in Equation~\ref{bltp_9}. 

\begin{equation}
\begin{array}{lr}
\underset{\mathbf{x}_u}{\text{Maximize}}\hspace{2mm} F_u(\mathbf{x}_u, \mathbf{ x}_l)=-2x_u+11x_l\\


\hspace{9mm} \underset{\mathbf{x}_l} {\text{Maximize}}\hspace{2mm} f_l(\mathbf{x}_u, \mathbf{ x}_l)=-x_u-3x_l \\
\hspace{25mm}s.t\hspace{2mm} x_u-2x_l\leq 4\\
\hspace{25mm} 2x_u-x_l\leq 24\\
\hspace{25mm} 3x_u+4x_l\leq 96\\
\hspace{25mm} x_u+4x_l\leq 126\\
\hspace{25mm} -4x_u+5x_l\leq 65\\
\hspace{25mm} x_u+4x_l\geq 8\\
\hspace{25mm} \mathbf{x}_u \in {[0,18]},\mathbf{x}_l \in [0,11] \\
%\hspace{25mm} \mathbf{x}_u, \mathbf{x}_l \leq 0\\
\end{array}
\label{bltp_9}
\end{equation}

The optimum values $F_u^*$= 85.0909and  $f_l^*$=-50.1818 are reported in ~\cite{Angelobilevel}. The corresponding optimum variables are $(\mathbf{x}^*_u, \mathbf{x}^*_l)=(17.4545,10.9091)$.


\subsection{BLTP10}

This problem is taken from~\cite{rajesh2003tabu}. The formulation is described in Equation~\ref{bltp_10}.

\begin{equation}
\begin{array}{lr}
\underset{\mathbf{x}_u}{\text{Minimize}}\hspace{2mm} F_u(\mathbf{x}_u, \mathbf{ x}_l)=(x_u-3)^2+(x_l-2)^2\\


\hspace{9mm} \underset{\mathbf{x}_l} {\text{Minimize}}\hspace{2mm} f_l(\mathbf{x}_u, \mathbf{ x}_l)=(x_l-5)^2 \\
\hspace{25mm}s.t\hspace{2mm} 2x_u-x_l\geq -1\\
\hspace{25mm}  -x_u+2x_l \geq 2\\
\hspace{25mm}  -x_u-2x_l \geq -14\\
\hspace{25mm} \mathbf{x}_u \in {[0,8]},\mathbf{x}_l \in [0,40] \\
%\hspace{25mm} \mathbf{x}_u \in {[0,8]},\mathbf{x}_l \geq 0 \\
\end{array}
\label{bltp_10}
\end{equation}
The optimum values $F_u^*$= 5 and  $f_l^*$=4 are reported in~\cite{Angelobilevel}. The corresponding optimum variables are $(\mathbf{x}^*_u, \mathbf{x}^*_l)=(1,3)$.

\subsection{BLTP11}

This problem is taken from~\cite{rajesh2003tabu}. The formulation is described in Equation~\ref{bltp_11}.

\begin{equation}
\begin{array}{lr}
\underset{\mathbf{x}_u}{\text{Minimize}}\hspace{2mm} F_u(\mathbf{x}_u, \mathbf{ x}_l)=(x_u-3)^2+(x_l-2)^2\\

\hspace{16mm}s.t\hspace{2mm} 2x_u-x_l\geq -1\\
\hspace{16mm}  -x_u+2x_l \geq 2\\
\hspace{16mm}  -x_u-2x_l \geq -14\\
\hspace{9mm} \underset{\mathbf{x}_l} {\text{Minimize}}\hspace{2mm} f_l(\mathbf{x}_u, \mathbf{ x}_l)=(x_l-5)^2 \\

\hspace{25mm} \mathbf{x}_u \in {[0,8]},\mathbf{x}_l \in [0,50] \\
%\hspace{25mm} \mathbf{x}_u \in {[0,8]},\mathbf{x}_l \geq 0 \\
\end{array}
\label{bltp_11}
\end{equation}

The optimum values $F_u^*$= 9 and  $f_l^*$=0 are reported in~\cite{Angelobilevel}. The corresponding optimum variables are $(\mathbf{x}^*_u, \mathbf{x}^*_l)=(3,5)$.

\subsection{BLTP12}

This problem is taken from~\cite{bard1982explicit}. The formulation is described in Equation~\ref{bltp_12}.

\begin{equation}
\begin{array}{lr}
\underset{\mathbf{x}_u}{\text{Maximize}}\hspace{2mm} F_u(\mathbf{x}_u, \mathbf{ x}_l)=2x_{u1}-x_{u2}-0.5x_{l1}\\
\hspace{9mm} \underset{\mathbf{x}_l} {\text{Maximize}}\hspace{2mm}f_l(\mathbf{x}_u, \mathbf{x}_l)=-x_{u1}-x_{u2}+4x_{l1}-x_{l2}\\

\hspace{25mm}s.t\hspace{2mm} 2x_{u1}-x_{l1}+x_{l2}\geq 2.5\\

\hspace{25mm}  -x_{u1}+3x_{u2}-x_{l2} \geq -2\\
\hspace{25mm}  -x_{u1}-x_{u2} \geq -2\\
\hspace{25mm} \mathbf{x}_u \in {[0,2]},\mathbf{x}_l \in [0,2] \\
%\hspace{25mm} \mathbf{x}_u, \mathbf{x}_l \geq 0 \\
\end{array}
\label{bltp_12}
\end{equation}

The best known values $F_u^*$= 3.25 and  $f_l^*$=4 are reported in~\cite{Angelobilevel}. The corresponding variables are $(\mathbf{x}^*_u, \mathbf{x}^*_l)=(2,0,1.5,0)$.

\subsection{BLTP13}

This problem is taken from~\cite{candler1982linear}. The formulation is described in Equation~\ref{bltp_13}.

\begin{equation}
\begin{array}{lr}
\underset{\mathbf{x}_u}{\text{Maximize}}\hspace{2mm} F_u(\mathbf{x}_u, \mathbf{ x}_l)=8x_{u1}+4x_{u2}-4x_{l1}+40x_{l2}+4x_{l3}\\


\hspace{9mm} \underset{\mathbf{x}_l} {\text{Maximize}}\hspace{2mm} f_l(\mathbf{x}_u, \mathbf{ x}_l)=-x_{u1}-2x_{u2}-x_{l1}-x_{l2}-2x_{l3} \\
\hspace{25mm}s.t\hspace{2mm} -x_l1+x_l2+x_l3\leq 1\\
\hspace{25mm} 2x_u1-x_l1+2x_l2-0.5x_l3\leq 1\\
\hspace{25mm} 2x_{u2}+2x_{l1}-x_{l2}-0.5x_{l3}\leq 1\\
\hspace{25mm} \mathbf{x}_u \in {[0,1]},\mathbf{x}_l \in [0,1] \\
%\hspace{25mm} \mathbf{x}_u, \mathbf{x}_l \geq 0\\
\end{array}
\label{bltp_13}
\end{equation}

The optimum values $F_u^*$= 29.2 and  $f_l^*$=-3.2 are reported in~\cite{candler1982linear}. The corresponding variables are $(\mathbf{x}^*_u, \mathbf{x}^*_l)=(0,0.9,0,0.6,0.4).$
