\chapter*{Abstract}
\pagestyle{plain}
%\addcontentsline{toc}{chapter}{\protect\numberline{}{Abstract}}
\addtotoc{Abstract}


Bilevel optimization, also referred to as bilevel programming, involves solving an upper level problem subject to the optimality of a corresponding lower level problem. The upper and lower level problems are also referred to as the \textit{leader} and \textit{follower} problems, respectively. Both levels have their associated objective(s), variable(s) and constraint(s). Such problems model real-life scenarios of cases where the performance of an upper level authority is realizable/sustainable only if the corresponding lower level objective is optimum. A number of practical applications in the field of engineering, logistics, economics and transportation have inherent nested structure that are suited to this type of modeling. The range of applications as well as a rapid increase in the size and complexity of such problems has prompted active interest in the design of efficient algorithms for bilevel optimization. 

Bilevel optimization problems present a number of unique and interesting challenges to algorithm design. The nested nature of the problem requires optimization of a lower level problem to evaluate \emph{each} upper level solution, which makes it computationally exhorbitant. Theoretically, an upper level solution is considered valid/feasible only if the corresponding lower level variables are the true global optimum of the lower level problem. Global optimality can be reliably asserted in very limited cases, for example convex and linear problems. In deceptive cases, an inaccurate lower level optimum may result in an objective value better than true optimum at the upper level, which poses a severe challenge for ranking/selection strategies used within any optimization technique. In turn, this also makes the performance evaluation very difficult since the performance cannot be judged based on the objective values alone. 

While the area of bilevel~(or more generally, multilevel) programming itself is not very new, most efforts in this direction up until about a decade ago considered solving linear or at most quadratic problems at both levels. Correspondingly, the focus on was on development of exact methods to solve such problems. However, such methods typically require assumptions on mathematical properties, which may not always hold in practical applications. With increasing use of computer simulation based evaluations in a number of disciplines in science and engineering, there is more need than ever to handle problems that are highly nonlinear or even black-box in nature. Metaheuristic algorithms, such as evolutionary algorithms are more suited to this emerging paradigm. The foray of evolutionary algorithms in bilevel programming is relatively recent and there remains scope of substantial development in the field in terms of addressing the aforementioned challenges.  

The work presented in this thesis is directed towards improving evolutionary techniques to enable them solve generic bilevel problems more accurately using lower number of function evaluations compared to the existing methods. Three key approaches are investigated towards accomplishing this: (a) effective hybridization of global and local search methods during different stages of the overall search; (b) use of surrogate models to guide the search using approximations in lieu of true function evaluations, and (c) use of a non-nested re-formulation of the problem. While most of the work is focused on single-objective problems, preliminary studies are also presened on multi-objective bilevel problems. The performance of the proposed approaches is evaluated on a comprehensive suite of mathematical test problems available in the literature, as well as some practical problems. The proposed approaches are observed to achieve a favourable balance between accuracy and computational expense for solving bilevel optimization problems, and thus exhibit suitability for use in real-life applications. 

%By nature bilevel problems also prove themselves as expensive optimization problems. My aim in this thesis is to produce some efficient algorithms which can handle different types of bilevel optimization problems such as single objective which contains single objective at both upper and lower level, multi follower problems which contains single at upper and multiple at lower and multi objective where both level contains two objectives. My first contribution to this thesis is a memetic algorithm to solve single objective bilevel problems. The idea is instead of only evolutionary algorithm, I applied evolutionary algorithm at the initial phase and local search at the later in the lower level problem. The upper level search also used local search at the end of global search for it's quick convergence ability. The method also used re-evaluation mechanism to avoid the local optima of the lower level solution only after the phase of global and local search at the upper level. The method performs superior performance than the existing extablished algorithm NBLEA, BLEAQ, BIDE and others in terms of both computational cost and convergence rate. This algorithm also applied to the multi follower optimization problems which also shown significant ability to solve these problems. The second proposed algorithm is based on surrogate which reduced the computational cost than BLMA significantly. For each upper level solution, I used multi surrogate model which reduced the cost and improved the quality of the lower level solution. Best solution at upper level has been re-evaluated by using special mechanism which contains both local and global search. Global search is applied first and then local search improved the solution quality with very good starting point. This algorithm performs better than the all existing stat of the art like BLMA, NBLEA, BLEAQ BIDE(true) and BIDE(surrogate) and others. The third algorithm is based on the incorporation of non-nested mechanism to the bilevel problems. The upper and lower level population are combined in this algorithm. It uses complementary constrained for the upper level problem if current lower level solution is not better than the neighboring lower level solutions. This algorithm also used re-evaluation mechanism like surrogate assisted approach and nested local search within the interval of ten generations. This approach is also perform better for the standard SMD problems suite. Finally, The thesis is concluded to my preliminary work for solving bilevel multi objective problems. The framework used nested differential algorithm for solving different types of bilevel problems. The algorithm is good for lower dimensional problems. However, it is not working for higher dimensional problems~(especially when it exceeds six). Overall my contributions in this thesis play significant role to solve bilevel optimization problems. It will also give a direction to the future bilevel researchers to work on more efficient approaches. In addition, all of our proposed approaches show significant convergence ability with reduced cost.   


